\documentclass[10pt,twocolumn]{article}
\usepackage[utf8]{inputenc}
\usepackage{tabto}
\usepackage{textcomp}
\usepackage{geometry}
\usepackage{indentfirst}
\usepackage{anyfontsize}
\usepackage{ragged2e}
\usepackage[czech]{babel}
 \geometry{
 a4paper,
 total={170mm,230mm},
 left=20mm,
 top=20mm,
 }
 
\setlength{\parskip}{0mm}

\title{Typografie a publikování ~~1. projekt}
\author{Richard Klem (xklemr00@stud.fit.vutbr.cz)}
\date{}


\begin{document}
\maketitle

\section{Hladká sazba}
\noindent Hladká sazba používá jeden stupeň, druh a řez písma\break
a je sázena na stanovenou šířku odstavce. Skládá\break
se z odstavců, obvykle začínajících zarážkou, nejde-li\break
o první odstavec za nadpisem. Mohou ale být sázeny\break
i bez zarážky\,–\,rozhodující je celková grafická úprava.\break
Odstavec končí východovou řádkou. Věty nesmějí za-čínat číslicí.


Zvýraznění barvou, podtržením, ani změnou písma\break
se v odstavcích nepoužívá. Hladká sazba je určena pře-devším
pro delší texty, jako je beletrie. Porušení konzis-tence
sazby působí v textu rušivě a unavuje čtenářův\break
zrak.
\section{Smíšená sazba}
\noindent Smíšená sazba má o něco volnější pravidla. Klasická\break
hladká sazba se doplňuje o další řezy písma pro zvý-\break raznění
důležitých pojmů. Existuje „pravidlo“:
\begin{quotation}
{{\fontfamily{qag}\selectfont{\large{Čím více druhů,}}} {\scriptsize{\MakeUppercase{řezů}}}, {\tiny{velikostí}}, barev
písma \underline{a jiných efektů} použijeme, tím \textit{{\large{profesionálněji}}}
bude dokument vypadat. Čtenář
tím bude vždy {\scriptsize{\MakeUppercase{nadšen!}}}}    
\end{quotation}


{\footnotesize{Tímto}} pravidlem se nikdy \textbf{nesmíte} řídit. Příliš časté
{\large{zvýrazňování}} textových elementů a změny velikosti
{\fontsize{20}{0}\selectfont písma} jsou známkou amatérismu autora a působí
{\fontsize{24}{0}\selectfont velmi} rušivě. Dobře navržený dokument nemá
obsahovat více než 4 řezy či druhy písma. Dobře navr-\breakžený
dokument je decentní, ne chaotický.


Důležitým znakem správně vysázeného dokumentu
je konzistence–například \textbf{tučný řez} písma bude vy-\break hrazen
pouze pro klíčová slova, \textit{skloněný řez} pouze\break
pro doposud neznámé pojmy a nebude se to míchat.\break
Skloněný řez nepůsobí tak rušivě a používá se častěji.\break
V \LaTeX u jej sázíme raději příkazem {\fontfamily{lmtt}\selectfont \textbackslash{}emph\{text\}}než\break
{\fontfamily{lmtt}\selectfont \textbackslash textit\{text\}}.


Smíšená sazba se nejčastěji používá pro sazbu vědec-\break kých
článků a technických zpráv. U delších dokumentů\break
vědeckého či technického charakteru je zvykem vysvět-\break lit
význam různých typů zvýraznění v úvodní kapitole.
\section{České odlišnosti}
\noindent Česká sazba se oproti okolnímu světu v některých\break
aspektech mírně liší. Jednou z odlišností je sazba uvo-\break zovek.
Uvozovky se v češtině používají převážně pro\break
zobrazení přímé řeči, zvýraznění přezdívek a ironie.\break
V češtině se používají uvozovky typu \uv{9966} místo “an-\break glických”
uvozovek nebo "dvojitých"uvozovek. Lze je\break
sázet připravenými příkazy nebo při použití UTF-8 kó-\break dování
i přímo.\par
Ve smíšené sazbě se řez uvozovek řídí řezem prvního\break
uvozovaného slova. Pokud je uvozována celá věta, sází\break
se koncová tečka před uvozovku, pokud se uvozuje slovo
nebo část věty, sází se tečka za uvozovku.\par
Druhou odlišností je pravidlo pro sázení konců\break
řádků. V české sazbě by řádek neměl končit osamoce-\break nou jednopísmennou
předložkou nebo spojkou. Spoj-\break
kou \uv{a} končit může pouze při sazbě do šířky 25 li-\break
ter. Abychom \LaTeX u zabránili v sázení osamocených\break
předložek, spojujeme je s následujícím slovem \emph{nezlo-\break
mitelnou mezerou}. Tu sázíme pomocí znaku \textbf{\textasciitilde}\ (vlnka,\break
tilda). Pro systematické doplnění vlnek slouží volně ši-\break
řitelný program \textit{vlna} od pana Olšáka\footnotemark[1].\par
Balíček {\fontfamily{lmtt}\selectfont fontenc} slouží ke korektnímu kódovaní\break
znaků s diakritikou, aby bylo možno v textu vyhledávat\break
a kopírovat z něj.
\section{Závěr}
\noindent Tento dokument\footnotemark[2] je členěn na sekce pomocí příkazu\break
\textbackslash {\fontfamily{lmtt}\selectfont
section}. Jedna ze sekcí schválně obsahuje několik\break
typografických prohřešků. V kontextu celého textu je\break
jistě snadno najdete. Je dobré znát možnosti \LaTeX u,\break
ale je také nutné vědět, kdy je nepoužít.

\footnotetext[1]{
Viz http://petr.olsak.net/ftp/olsak/vlna/}
\footnotetext[2]{Příliš mnoho poznámek pod čarou čtenáře zbytečně rozpty-\break luje.
Používejte je opravdu střídmě. (Šetřete i s textem v závor-\break kách.)}


\end{document}
