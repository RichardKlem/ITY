\documentclass[a4paper, 11pt]{article}

\usepackage[top=3cm, left=2cm, text={17cm,24cm}]{geometry}
\usepackage[utf8]{inputenc}
\usepackage[czech]{babel}
\usepackage{times}
\usepackage{multirow}
\usepackage{ragged2e}
\usepackage{csquotes}
\usepackage{url}
\DeclareUrlCommand\url{\def\UrlLeft{<}\def\UrlRight{>} \urlstyle{tt}}

\begin{document}
\begin{titlepage}
\begin{center}
\textsc{\Huge{Vysoké učení technické v Brně\\[0,3em]}
{\huge{Fakulta informačních technologií}}}\\
\vspace{\stretch{0.382}}
{\LARGE Typografie a publikování -- 4. projekt}\\[0,3em]
{\Huge{Typografie a citace}}\\
\vspace{\stretch{0.618}}
{\Large \today \hfill Richard Klem}
\end{center}
\end{titlepage}


\section{Úvod}
\justifying
Na začátek si něco povíme o typografii, a proč je dobré o této vědní disciplíně něco vědět. V druhé části si řekneme něco k citování.
\section{Typografie}
Nejprve by bylo dobré pochopit, co to tygrafie vlastně je.

\begin{quote}
 \emph{\uv{Typografie je disciplína zabývající se písmem, především jeho správným výběrem, použitím a sazbou. Cílem typografie je zajistit čtenáři snazší čtení, efektivnější vnímání čteného textu a případně i vyloučit možné chyby a nejednoznačnosti plynoucí z více možných zápisů téže věty.}}\cite{Typografie}
\end{quote}
Tento text je sázen s pomocí \LaTeX u, což je nástroj na základu \TeX u. \TeX vytvořil pan profesor Knuth ze Stanfordské univerzite poté, co nebyl spokojen s tehdejšími sázecími nastroji.\cite{Olsak1999}
A my mu za tento nástroj vděčíme dodnes.
Při procházení internetu jsem narazil na web. stránce \emph{brainpickings.org}\cite{brainpicking} na zajímavý soupis deseti děl z oblasti typografie, které stojí za přečtení. Z nich mě zaujal titul s názvem \emph{Cultural Connectives}, v kterém autor rozebírá a porovnává rozdíly, ale i společné body, latinky a arabského písma. Mimo jiné se zde dozvíte, že zatímco v latince se slova do bloku sází pomocí mezer mezi slovy a písmeny nebo pomocí pomlček a půlení slova přes dva řádky, v arabském písmu se slova spojují horizontálními linkami.\cite{CulturalConnectives}

\subsection{Typografie ve vysokoškolských pracích}
Znalosti typografie se určitě hodí při práci na bakalářské nebo diplomové práci. Dříve se považovali práce vysázené například pomocé \LaTeX u za zdařilejší(z hlediska typografie) než ty napsané například v programu MS Office Word.

Nyní se již tyto rozdíly ztenčují díky vyvýjejícímu se MS Office Word, nicméně stále tento trend přetrvává. Poněkud mě zarazilo, když jsem v bakalářské práci\cite{EliskaCvingrafova2011}, která se věnuje částečně i typografii, nenašel ani jednu zmínku o \LaTeX u. Naopak Martin Černý se ve své diplomové práci věnuje podrobně znakovým sadám, fontům, písmům apod. v typografických systémech, včetně \TeX u.\cite{MartinCerny1999}

\section{Citování}
V každém našem díle čerpáme z různých zdrojů. Je dobré tyto zdroje správně citovat, abychom poskytli čtenáři možnost ověřit si informace, rozšířit si znalosti apod.

Mimochodem, věděli jste, že hnědí trpaslíci jsou vesmírná tělěsa velkosti mezi velkou planetou a malou hvězdou? Jejich atmosféra má podobné vlastnosti jako ta velkých plynných planet.\cite{Science1}
Když člověk začne přemýšlet o vesmíru, často se mu z toho úplně zavaří hlava.
O tom, co se děje v našem mozku víme mnoho, ale zdaleka ne vše. Věda jde ale stále kupředu.
Nedávno vědci objevili novou reakci probíhající v našem mozku mezi dopaminem a chromatinem. Tyto dvě látky se spojují, a provádí doposud neznámou formu epigenetické regulace zvanou dopaminylace.\cite{Science2}

Tyto a mnoho dalších zajímavých faktů ze světa vědy se lze dočíst v časopisech \emph{Science}\cite{ScienceJournal}, které vycházejí už od roku 1880.

Od března roku 2011 je platná nová citační norma ČSN ISO 690. Norma má svoji vlastní webovou stránku, kde se můžete dočíst přesná pravidla a podívat se na přílady správného citování.\cite{csniso690}

\newpage
\bibliographystyle{czechiso}
\bibliography{zdroje}
\end{document}