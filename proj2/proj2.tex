\documentclass[11pt, twocolumn]{article}
\usepackage[utf8]{inputenc}
\usepackage[IL2]{fontenc}
\usepackage{amsthm, amssymb, amsmath}
\usepackage{times}
\usepackage{indentfirst}
\usepackage{geometry}
\usepackage[czech]{babel}
\geometry{
a4paper,
total={180mm,250mm},
left=15mm,
top=25mm,
}

\newtheorem{theorem}{Definice}
\newtheorem{sentence}{Věta}

\begin{document}
\begin{titlepage}
\begin{center}
\textsc{\Huge{
Fakulta informačních technologií\\[0.4em]
Vysoké učení technické v Brně}}\\
\vspace{\stretch{0.382}}
{\LARGE Typografie a publikování -- 2. projekt\\[0.3em]
Sazba dokumentů a matematických výrazů}\\
\vspace{\stretch{0.618}}
{\Large 2020 \hfill Richard Klem(xklemr00)}
\end{center}
\end{titlepage}
\section*{Úvod}
\noindent V této úloze si vyzkoušíme sazbu titulní strany, matematic-\break kých vzorců, prostředí a dalších textových struktur obvyklých
pro technicky zaměřené texty (například rovnice (\ref{eq_2}) nebo Definice \ref{def_2} na straně \pageref{def_2}). Pro vytvoření těchto odkazů
používáme příkazy 
\verb|\label|, \verb|\ref| a \verb|\pageref|.

Na titulní straně je využito sázení nadpisu podle optického
středu s využitím zlatého řezu. Tento postup byl
probírán na přednášce. Dále je použito odřádkování se
zadanou relativní velikostí 0.4em a 0.3em.
\section{Matematický text}
\noindent Nejprve se podíváme na sázení matematických symbolů
a~výrazů v plynulém textu včetně sazby definic a vět s využitím
balíku \verb|amsthm|. Rovněž použijeme poznámku pod
čarou s použitím příkazu \verb|\footnote|. Někdy je vhodné
použít konstrukci \verb|${}$| nebo \verb|\mbox{}|  která říká, že
(matematický) text nemá být zalomen. V následující definici
je nastavena mezera mezi jednotlivými položkami
\verb|\item| na 0.05em.

\begin{theorem}
\label{def_1}
\emph{
Turingův stroj \emph{(TS)} \emph{je definován jako šestice
tvaru} \(M =(Q, \Sigma, \Gamma, \delta, q_0, q_F)\),\emph{ kde:}}
\begin{itemize}
\setlength\itemsep{0.05em}
\item \emph{\(Q\)\emph{ je konečná množina} vnitřních (řídicích) stavů,}
\item \emph{\(\Sigma\) \emph{je konečná množina symbolů nazývaná} vstupní
abeceda, \(\Delta\) \(\notin\) \(\Sigma\),}
\item \emph{\(\Gamma\) \emph{je konečná množina symbolů, \(\Sigma\subset\Gamma\), \(\Delta\in\Gamma\),
nazývaná} pásková abeceda,}
\item \emph{\(\delta\) : (\(Q\setminus\{q_F\}\))\(\times\)\(\Gamma\) \(\rightarrow\) \(Q\)\(\times\)(\(\Gamma\)\(\cup\)\{\(L, R\)\}, \emph{kde L, R \(\notin\) \(\Gamma\), je parciální} přechodová funkce, \emph{a}}
\item \emph{\(q_0 \in Q\) \emph{je} počáteční stav \emph{a} \(q_f \in Q\) \emph{je} koncový stav.}
\end{itemize}

\emph{Symbol \(\Delta\) značí tzv. \emph{blank} (prázdný symbol), který se vyskytuje na místech pásky, která nebyla ještě použita.}

\emph{
\emph{Konfigurace pásky} se skládá z nekonečného řetězce,
který reprezentuje obsah pásky a pozice hlavy na tomto
řetězci. Jedná se o prvek množiny
\mbox{\(\{\gamma \Delta^{\omega}\)
\(|\)
\(\gamma \in \Gamma^{*}\}\times\mathbb{N}^{1}\)}.
\emph{Konfiguraci pásky} obvykle zapisujeme jako \(\Delta xyz\underline{z}x\Delta...\)
(podtržení značí pozici hlavy). \emph{Konfigurace stroje} je pak
dána stavem řízení a konfigurací pásky. Formálně se jedná
o prvek množiny \mbox{\(Q\times\{\gamma \Delta^{\omega}\) \(|\) \(\gamma \in \Gamma^{*}\}\times\mathbb{N}\)}.
}
\end{theorem}
\subsection{Podsekce obsahující větu a odkaz}
\begin{theorem}
\label{def_2}
\emph{Řetězec \(w\) nad abecedou \(\Sigma\) je přijat TS \emph{\(M\)
jestliže \(M\) při aktivaci z počáteční konfigurace pásky}
\footnotetext[1]{Pro libovolnou abecedu \(\Sigma\) je \(\Sigma^{\omega}\) množina všech nekonečných
řetězců nad \(\Sigma\), tj. nekonečných posloupností symbolů ze \(\Sigma\).}
\(\underline{\Delta}w\Delta...\)\emph{ a počátečního stavu \(q_0\) zastaví přechodem do koncového stavu \(q_F\), tj. (\(q_0, \Delta w\Delta^{\omega},0\)) \(\mathop{\vdash}\limits^{*}_{M}\) (\(q_F,\gamma,n\)) pro nějaké \(\gamma \in \Gamma^{*}\) a \(n \in \mathbb{N}\).}}

Množinu \(L(M) = \{w \) \(|\) \(w\) je přijat TS \(M\}\subseteq\Sigma^{*}\) nazý-\break váme jazyk přijímaný TS \(M\).
\end{theorem}
Nyní si vyzkoušíme sazbu vět a důkazů opět s použitím balíku \verb|amsthm|.
\begin{sentence}
Třída jazyků, které jsou přijímány TS, odpovídá \emph{rekurzivně vyčíslitelným jazykům.}
\end{sentence}
\begin{proof}
V důkaze vyjdeme z Definice \ref{def_1} a \ref{def_2}.
\end{proof}
\section{Rovnice}
\noindent Složitější matematické formulace sázíme mimo plynulý text. Lze umístit několik výrazů na jeden řádek, ale pak je třeba tyto vhodně oddělit, například příkazem \verb|\quad|.

\[\quad \sqrt[i]{x^3_i}\quad\text{kde }x_{i}\text{ je } i\text{-té sudé číslo}\quad y_{i}^{2\cdot y_{i}} \neq y_{i}^{y_{i}^{y_{i}}}\]

V rovnici (\ref{eq_1}) jsou využity tři typy závorek s různou
explicitně definovanou velikostí.
\begin{equation}
\label{eq_1}
x \quad = \quad \bigg\{\Big(\big[a+b\big]*c\Big)^{d}\oplus 1 \bigg\}
\end{equation}
\begin{equation}
\label{eq_2}
    y \quad=\quad \lim_{x\to\infty} \frac{\sin^2x+\cos^2x}{\frac{1}{\log_{10} x}}
\end{equation}



\begin{picture}(40,40)
    \linethickness{0.5pt}
    \put(0,0){\line(1,0){20}}
    \put(0,30){\line(1,0){20}}
    \put(0,0){\line(0,1){30}}
    \put(20,0){\line(0,1){30}}
    \put(5,5){\line(1,0){10}}
    \put(5,8){\line(1,0){10}}
    \put(5,16){\line(1,0){10}}
    \put(5,19){\line(1,0){10}}
    \put(5,22){\line(1,0){10}}
    \put(5,25){\line(1,0){10}}
\end{picture}


V této větě vidíme, jak vypadá implicitní vysázení limity
\(\lim_{n\to\infty} f(n)\) v normálním odstavci textu. Podobně
je to i s dalšími symboly jako \(\sum_{i=1}^{n}2^i\) či \(\bigcap_{A\in \mathcal{B}}A\). V pří-\break padě
vzorců \(\lim\limits_{n\to\infty} f(n)\) a \(\sum\limits_{i=1}^{n}2^i\) jsme si vynutili méně
úspornou sazbu příkazem \verb|\limits|.
\section{Matice}
\noindent Pro sázení matic se velmi často používá prostředí \verb|array| a závorky (\verb|\left|, \verb|\right|).
\[
\begin{pmatrix}
  
  a+b & \widehat{\xi+\omega} & \hat{\pi} \\
     \Vec{\mathbf{a}}& \overleftrightarrow{AC} & \beta

\end{pmatrix} = 1 \Longleftrightarrow \mathbb{Q} = \mathcal{R}\]
Prostředí \verb|array| lze úspěšně využít i jinde.
\[
\binom{n}{k}
=
\left\{
\begin{array}{cl}
    0 & \text{pro } k  < 0 \text{ nebo }k > n\\
    \frac{n!}{k!(n-k)!} & \text{pro }0 \leq k \leq n.
\end{array} 
\right.\]
\end{document}
